\documentclass[preprint,12pt]{elsarticle}
% \documentclass[draft,12pt]{elsarticle}

\usepackage{hyperref}
\usepackage{graphicx}
\usepackage{subcaption}
\usepackage{amssymb}
\usepackage{amsmath}
\usepackage{multirow}
\usepackage{relsize}
\usepackage[utf8]{inputenc}
\usepackage{cleveref}
\usepackage{algorithm}
\usepackage[noend]{algpseudocode}
\usepackage[section]{placeins}
\usepackage{booktabs}
\usepackage{url}

% For the TODOs
\usepackage{xcolor}
\usepackage{xargs}
\usepackage[colorinlistoftodos,textsize=footnotesize]{todonotes}
\newcommand{\todoin}{\todo[inline]}
% from here: https://tex.stackexchange.com/questions/9796/how-to-add-todo-notes
\newcommandx{\unsure}[2][1=]{\todo[linecolor=red,backgroundcolor=red!25,bordercolor=red,#1]{#2}}
\newcommandx{\change}[2][1=]{\todo[linecolor=blue,backgroundcolor=blue!25,bordercolor=blue,#1]{#2}}
\newcommandx{\info}[2][1=]{\todo[linecolor=OliveGreen,backgroundcolor=OliveGreen!25,bordercolor=OliveGreen,#1]{#2}}

%Boldtype for greek symbols
\newcommand{\teng}[1]{\ensuremath{\boldsymbol{#1}}}
\newcommand{\ten}[1]{\ensuremath{\mathbf{#1}}}

\usepackage{lineno}
% \linenumbers

\journal{}

\begin{document}

\begin{frontmatter}

  \title{}
  \author[XXX]{Dinesh Adepu\corref{cor1}}
  \ead{d.dinesh@surrey.ac.uk}
  \author[XXX]{Pawan Negi \corref{cor1}}
  \ead{xxx}
  \author[University of Surrey]{Chuan Yu Wu}
  \ead{XXX}
\address[xxx]{xxx}

\cortext[cor1]{Corresponding author}


\begin{abstract}
\end{abstract}

\begin{keyword}
%% keywords here, in the form: keyword \sep keyword
{xxx}, {xxx}, {xxx}

%% MSC codes here, in the form: \MSC code \sep code
%% or \MSC[2008] code \sep code (2000 is the default)

\end{keyword}

\end{frontmatter}

% \linenumbers

\section{Internal comments}
\label{sec:intro}
Use moving least square IBM single particle layer for rigid fluid coupling.

\section{Introduction}
\label{sec:intro}



\FloatBarrier%
\section{Results}
\label{sec:results}


\FloatBarrier%
\subsection{Plane poiseuille flow}
\label{sec:res:ppf}

\begin{figure}[!htpb]
  \centering
  \includegraphics[width=0.6\textwidth]{figures/plane_poiseuille_flow_2D/case_1/comparison}
  \caption{Velocity comparison for a 2D plane Poiseuille flow}
\label{fig:plane_poiseuille_flow_2D-case_1-comparison}
\end{figure}

% \FloatBarrier%
% \subsubsection{A steady hydrostatic tank}
% First we need a steady hydrostatic tank with no particles oscillating and a no
% pressure oscillations. I have created a files with name
% $dinesh\_2023\_steady\_hs\_tank.py$. We will run it for a total time of $0.3$
% seconds.



\FloatBarrier%
\subsection{Circular body falling in a steady tank}
\label{sec:res:circular-body-entry}

% file corresponding to this example is
% skillen_2013_circular_water_entry.py

\begin{figure}[!htpb]
  \centering
  \includegraphics[width=0.6\textwidth]{figures/skillen_2013_water_entry_half_buoyant/penetration_vs_t}
  \caption{Water entry half buoyant body}
\label{fig:xxxx}
\end{figure}

\begin{figure}[!htpb]
  \centering
  \includegraphics[width=0.6\textwidth]{figures/skillen_2013_water_entry_neutrally_buoyant/penetration_vs_t}
  \caption{Water entry neutrally buoyant body}
\label{fig:xxxx}
\end{figure}

\citet{skillen_incompressible_2013}


\FloatBarrier%
\subsection{2D water wedge entry}
\label{sec:res:wedge_entry}

\citet{sun_numerical_2015}

\begin{figure}[!htpb]
  \centering
  \includegraphics[width=0.6\textwidth]{figures/sun_2015_2d_water_wedge_entry/velocity_vs_t}
  \caption{Velocity of the 2d water wedge entry problem}
\label{fig:xxxx}
\end{figure}
% file corresponding to this example is
% skillen_2013_forced_wedge_water_entry.py


\FloatBarrier%
\subsection{Two cylinders in a shear flow}
\label{sec:res:hpf}

% file corresponding to this example is
% ng_2021_3_1_1_two_cylinders_in_shear_flow.py
\citet{ng_numerical_2021}

\begin{figure}[!htpb]
  \centering
  \includegraphics[width=0.6\textwidth]{figures/ng_2021_two_cylinders_in_shear_flow/u_cm_vs_t}
  \caption{Two cylinders in a parallel plate flow.}
\label{fig:xxx}
\end{figure}


\FloatBarrier%
\subsection{Falling of a circular cylinder in a closed channel}
\label{sec:res:hpf}

% file corresponding to this example is
% hashemi_2012_falling_circular_cylinder_in_closed_channel.py

\citet{hashemi_modified_2012}

\begin{figure}[!htpb]
  \centering
  \includegraphics[width=0.6\textwidth]{figures/hashemi_2012_falling_circular_cylinder_in_closed_channel/velocity_vs_t}
  \caption{Velocity vs time of circular body falling in a closed tank}
\label{fig:xxxx}
\end{figure}

\begin{figure}[!htpb]
  \centering
  \includegraphics[width=0.6\textwidth]{figures/hashemi_2012_falling_circular_cylinder_in_closed_channel/position_vs_t}
  \caption{Position vs time of circular body falling in a closed tank}
\label{fig:xxxx}
\end{figure}


\FloatBarrier%
\subsection{A neutrally buoyant circular cylinder in a shear flow}
\label{sec:res:hpf}
% file corresponding to this example is
% hashemi_2012_neutrally_buoyant_circular_cylinder_in_shear_flow.py
\citet{hashemi_modified_2012}

\ten{TODO}

% \begin{figure}[!htpb]
%   \centering
%   \includegraphics[width=0.6\textwidth]{figures/plane_poiseuille_flow_2D/case_1/comparison}
%   \caption{Velocity comparison for a 2D plane Poiseuille flow}
% \label{fig:plane_poiseuille_flow_2D-case_1-comparison}
% \end{figure}


\FloatBarrier%
\section{Conclusions}
\label{sec:conclusions}


\section*{References}


\bibliographystyle{model6-num-names}
\bibliography{references}
\end{document}

% ============================
% Table template for reference
% ============================
% \begin{table}[!ht]
%   \centering
%   \begin{tabular}[!ht]{ll}
%     \toprule
%     Quantity & Values\\
%     \midrule
%     $L$, length of the domain & 1 m \\
%     Time of simulation & 2.5 s \\
%     $c_s$ & 10 m/s \\
%     $\rho_0$, reference density & 1 kg/m\textsuperscript{3} \\
%     Reynolds number & 200 \& 1000 \\
%     Resolution, $L/\Delta x_{\max} : L/\Delta x_{\min}$ & $[100:200]$ \& $[150:300]$\\
%     Smoothing length factor, $h/\Delta x$ & 1.0\\
%     \bottomrule
%   \end{tabular}
%   \caption{Parameters used for the Taylor-Green vortex problem.}%
%   \label{tab:tgv-params}
% \end{table}

%%% Local Variables:
%%% mode: latex
%%% TeX-master: "paper"
%%% fill-column: 78
%%% End:
