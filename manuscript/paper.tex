\documentclass[preprint,12pt]{elsarticle}
% \documentclass[draft,12pt]{elsarticle}

\usepackage{hyperref}
\usepackage{graphicx}
\usepackage{subcaption}
\usepackage{amssymb}
\usepackage{amsmath}
\usepackage{multirow}
\usepackage{relsize}
\usepackage[utf8]{inputenc}
\usepackage{cleveref}
\usepackage{algorithm}
\usepackage[noend]{algpseudocode}
\usepackage[section]{placeins}
\usepackage{booktabs}
\usepackage{url}

% For the TODOs
\usepackage{xcolor}
\usepackage{xargs}
\usepackage[colorinlistoftodos,textsize=footnotesize]{todonotes}
\newcommand{\todoin}{\todo[inline]}
% from here: https://tex.stackexchange.com/questions/9796/how-to-add-todo-notes
\newcommandx{\unsure}[2][1=]{\todo[linecolor=red,backgroundcolor=red!25,bordercolor=red,#1]{#2}}
\newcommandx{\change}[2][1=]{\todo[linecolor=blue,backgroundcolor=blue!25,bordercolor=blue,#1]{#2}}
\newcommandx{\info}[2][1=]{\todo[linecolor=OliveGreen,backgroundcolor=OliveGreen!25,bordercolor=OliveGreen,#1]{#2}}

%Boldtype for greek symbols
\newcommand{\teng}[1]{\ensuremath{\boldsymbol{#1}}}
\newcommand{\ten}[1]{\ensuremath{\mathbf{#1}}}

\usepackage{lineno}
\linenumbers

\journal{}

\begin{document}

\begin{frontmatter}

  \title{}
  \author[IITB]{Dinesh Adepu\corref{cor1}}
  \ead{d.dinesh@surrey.ac.uk}
  \author[XXX]{Pawan Negi \corref{cor1}}
  \ead{xxx}
  \author[University of Surrey]{Chuan Yu Wu}
  \ead{xyz@com}
\address[UoS]{Department of Aerospace Engineering, Indian Institute of
  Technology Bombay, Powai, Mumbai 400076}

\cortext[cor1]{Corresponding author}


\begin{abstract}

\end{abstract}

\begin{keyword}
%% keywords here, in the form: keyword \sep keyword
{Elastic solids collision}, {Frictional contact}, {SPH}, {Transport Velocity Formulation}

%% MSC codes here, in the form: \MSC code \sep code
%% or \MSC[2008] code \sep code (2000 is the default)

\end{keyword}

\end{frontmatter}

% \linenumbers

\section{Introduction}
\label{sec:intro}



\FloatBarrier%
\section{Results}
\label{sec:results}


\FloatBarrier%
\subsection{Plane poiseuille flow}
\label{sec:res:ppf}


\FloatBarrier%
\subsection{Hagen-Poiseuille flow}
\label{sec:res:hpf}



\FloatBarrier%
\subsection{A neutrally buoyant circular cylinder in a shear flow}
\label{sec:res:hpf}

\citet{hashemi_modified_2012}

\FloatBarrier%
\subsection{Falling of a circular cylinder in a closed channel}
\label{sec:res:hpf}

\citet{hashemi_modified_2012}

\FloatBarrier%
\subsection{Two interacting circular cylinders in a shear flow}
\label{sec:res:hpf}

\citet{hashemi_modified_2012}


\FloatBarrier%
\subsection{Falling of two circular cylinders in a closed channel}
\label{sec:res:hpf}

\citet{hashemi_modified_2012}




\FloatBarrier%
\section{Conclusions}
\label{sec:conclusions}

In this paper we have demonstrated a simple approach to effectively handle the
collision between elastic solids modeled using an updated Lagrangian SPH
model. A contact force model is used to handle the collision between bodies. A
surface aware spring based contact force is used to handle the collision
between bodies. This effectively allows us to model collision and friction
accurately. In addition this eliminates any spurious forces that are commonly
seen with SPH when two bodies are nearby but not in actual contact. The
contact force model utilized in the current work is sensitive towards the
primary and the secondary body chosen. A careful analysis is carried out to
understand which body is to be considered as primary among the colliding
solids, and it is found that choosing the body with the highest curvature as
the primary body gives the best results. Further, we have made our
implementation open-source.

It has been demonstrated that the current model is able to predict the post
collision behaviour of the colliding bodies by simulating collision between
flat, and curved interfaces in two and three dimensions. A sliding elastic
body is simulated to test the frictional part of the contact model. Finally,
the full scale model is applied to model the stress propagation in granular
discs for the first time in SPH. The results compare well with those of FEM as
well as analytical studies.

A non-linear contact force model can be implemented in the future work. The
current work can be easily extended to the modeling of collision between
elastic and elastic-plastic bodies. Also, the collision between the bodies
undergoing breakage can be easily captured with the current framework.

\section*{References}


\bibliographystyle{model6-num-names}
\bibliography{references}
\end{document}

% ============================
% Table template for reference
% ============================
% \begin{table}[!ht]
%   \centering
%   \begin{tabular}[!ht]{ll}
%     \toprule
%     Quantity & Values\\
%     \midrule
%     $L$, length of the domain & 1 m \\
%     Time of simulation & 2.5 s \\
%     $c_s$ & 10 m/s \\
%     $\rho_0$, reference density & 1 kg/m\textsuperscript{3} \\
%     Reynolds number & 200 \& 1000 \\
%     Resolution, $L/\Delta x_{\max} : L/\Delta x_{\min}$ & $[100:200]$ \& $[150:300]$\\
%     Smoothing length factor, $h/\Delta x$ & 1.0\\
%     \bottomrule
%   \end{tabular}
%   \caption{Parameters used for the Taylor-Green vortex problem.}%
%   \label{tab:tgv-params}
% \end{table}

%%% Local Variables:
%%% mode: latex
%%% TeX-master: "paper"
%%% fill-column: 78
%%% End:
